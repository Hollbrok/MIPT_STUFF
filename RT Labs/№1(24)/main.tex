\documentclass[a4paper, 12pt]{article}%тип документа

%%%Библиотеки
	%\usepackage[warn]{mathtext}	
	\usepackage[T2A]{fontenc} % кодировка
	\usepackage[utf8]{inputenc} % кодировка исходного текста
	\usepackage[english,russian]{babel} % локализация и переносы
	\usepackage{caption}
	\usepackage{listings}
	\usepackage{amsmath,amsfonts,amssymb,amsthm,mathtools}
	\usepackage{wasysym}
	\usepackage{graphicx}%Вставка картинок правильная
	\usepackage{float}%"Плавающие" картинки
	\usepackage{wrapfig}%Обтекание фигур (таблиц, картинок и прочего)
	\usepackage{fancyhdr} %загрузим пакет
	\usepackage{lscape}
	\usepackage{xcolor}
	\usepackage[normalem]{ulem}
	\usepackage{hyperref}

%%%Конец библиотек




%%%Настройка ссылок
	\hypersetup
	{
		colorlinks=true,
		linkcolor=blue,
		filecolor=magenta,
		urlcolor=blue
	}
%%%Конец настройки ссылок


%%%Настройка колонтитулы
	\pagestyle{fancy}
	\fancyhead{}
	\fancyhead[L]{Вопрос по выбору}
	\fancyhead[R]{Талашкевич Даниил, группа Б01-009}
	\fancyfoot[C]{\thepage}
%%%конец настройки колонтитулы



							\begin{document}
						%%%%Начало документа%%%%


%%%Начало титульника
\begin{titlepage}

	\newpage
	\begin{center}
		\normalsize Московский физико-технический институт \\(госудраственный 			университет)
	\end{center}

	\vspace{6em}

	\begin{center}
		\Large Лабораторная работа по РТ лабам\\
	\end{center}

	\vspace{1em}

	\begin{center}
		\large \textbf{Безынерционные линейные цепи [24]}
	\end{center}

	\vspace{2em}

	\begin{center}
		\large Талашкевич Даниил Александрович\\
		Группа Б01-009
	\end{center}

	\vspace{\fill}

	\begin{center}
	Долгопрудный \\2021
	\end{center}
	
\end{titlepage}
%%%Конец Титульника



%%%Настройка оглавления и нумерации страниц
	\thispagestyle{empty}
	\newpage
	\tableofcontents
	\newpage
	\setcounter{page}{1}
%%%Настройка оглавления и нумерации страниц


					%%%%%%Начало работы с текстом%%%%%%
					
\section{Делитель напряжения}

\subsection{Измерение $R^{*}$}

Собрали на макетной плане делитель напряжения, с $E^{*} = 2 B$, при напряжении питания $E = 10 B$. Резистор $R_1$ выбрали $7.5 $ кОм, тогда $R_2 = 1,87 \approx 1,8$ кОм (ближайшее значение сопротивления, которые было в лаборатории на момент выполнения). При заданных $R_1, R_2$ получаем $E^{*}_{\text{теор}} = 1,94 B$. Экспериментально было получено значение $1,98$ В.

$R_l$ был выбран 1кОм, отсюда, при полученном $U_l = 0,8 B$, следует значение для $R^{*} = 1,45$ кОм.					
\subsection{Оценка коэффициента передачи $K$}

Теперь задействуем генератор, а именно подадим синусоидальное напряжение $e$. Измерив эффективные значение $u, e$ получаем значение для $K = 0,193$. Посчитаем теоретическое значение коэффициента передачи $K_{\text{теор}} = \frac{R_2}{R_1 + R_2} = 0,194$. 

\section{Параллельный сумматор}
					
					
\end{document}
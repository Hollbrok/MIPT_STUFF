\documentclass[a4paper, 14pt]{extarticle}%тип документа
\date{}

%%%Библиотеки
	%\usepackage[warn]{mathtext}	
	\usepackage[T2A]{fontenc} % кодировка
	\usepackage[utf8]{inputenc} % кодировка исходного текста
	\usepackage[english,russian]{babel} % локализация и переносы
	\usepackage{caption}
	\usepackage{listings}
	\usepackage{amsmath,amsfonts,amssymb,amsthm,mathtools}
	\usepackage{wasysym}
	\usepackage{graphicx}%Вставка картинок правильная
	\usepackage{float}%"Плавающие" картинки
	\usepackage{wrapfig}%Обтекание фигур (таблиц, картинок и прочего)
	\usepackage{fancyhdr} %загрузим пакет
	\usepackage{lscape}
	\usepackage{xcolor}
	\usepackage[normalem]{ulem}
	\usepackage{hyperref}

%%%Конец библиотек


\usepackage{graphicx}
\usepackage{cmap}
\usepackage[T2A]{fontenc}
\usepackage[utf8]{inputenc}

%%\renewcommand{\footrulewidth}{ .0em }
\usepackage[english,russian]{babel}
\usepackage{multirow} % Слияние строк в таблице


%Русский язык
\usepackage[T2A]{fontenc} %кодировка
\usepackage[utf8]{inputenc} %кодировка исходного кода
\usepackage[english,russian]{babel} %локализация и переносы

%Таблицы
%\usepackage[table,xcdraw]{xcolor}
%\usepackage{booktabs}

%Математика
\usepackage{amsmath, amsfonts, amssymb, amsthm, mathtools}

%отступы
\usepackage[left=2cm,right=2cm,top=2cm,bottom=3cm,bindingoffset=0cm]{geometry}

%Вставка картинок
\usepackage{graphicx}
\usepackage{wrapfig, caption}
\graphicspath{}
\DeclareGraphicsExtensions{.pdf,.png,.jpg, .jpeg}
\newcommand\ECaption[1]{%
     \captionsetup{font=footnotesize}%
     \caption{#1}}

%Таблицы
%\usepackage[table,xcdraw]{xcolor}
%\usepackage{booktabs}

%Графики
\usepackage{pgfplots}
\pgfplotsset{compat=1.9}
\usepackage[english,russian]{babel}
\usepackage{multirow} % Слияние строк в таблице



\newcommand
{\un}[1]
{\ensuremath{\text{#1}}}

%%%Настройка ссылок
	\hypersetup
	{
		colorlinks=true,
		linkcolor=blue,
		filecolor=magenta,
		urlcolor=blue
	}
%%%Конец настройки ссылок

%%%Настройка колонтитулы
	\pagestyle{fancy}
	\fancyhead{}
	\fancyhead[L]{Лабораторная работа}
	\fancyhead[R]{Талашкевич Даниил, группа Б01-009}
	\fancyfoot[C]{\thepage}
%%%конец настройки колонтитулы\
\begin{document}

%%%Начало титульника
\begin{titlepage}

	\newpage
	\begin{center}
		\normalsize Московский физико-технический институт \\(госудраственный 			университет)
	\end{center}

	\vspace{6em}

	\begin{center}
		\Large Лабораторная работа по РТ цепям\\
	\end{center}

	\vspace{1em}

	\begin{center}
		\large \textbf{Длинный цепи [23]}
	\end{center}

	\vspace{2em}

	\begin{center}
		\large Талашкевич Даниил Александрович\\
		Группа Б01-009
	\end{center}

	\vspace{\fill}

	\begin{center}
	Долгопрудный \\2021
	\end{center}
	
\end{titlepage}
%%%Конец Титульника



%%%Настройка оглавления и нумерации страниц
	\thispagestyle{empty}
	\newpage
	\tableofcontents
	\newpage
	\setcounter{page}{1}
%%%Настройка оглавления и нумерации страниц


\section{Исследование параметров линии}

Так как нет возможности собрать плату и измерить её параметры, выполнение этого пункта пропускаю.

\section{Исследование переходных процессов}

Исследования проводим в режиме $Transient MicroCap$, с подготовленной моделью (файл $TLine.cir$), который содержит длинную линию с волновым сопротивлением $w = 50$ Ом, без потерь, время распространения $\tau=\frac{l}{w}=10$ нс. Линия питается от источника единичного перепада напряжения $V=1$ В.

Наблюдаются напряжения в узлах $e, u$ на входе и выходе линии (переменные $v(e), v(u))$ и входной/выходной токи $i(s) / i(l)$ через виртуальные резисторы $s, l$ с нулевыми сопротивлениями.

В этой модели (файле) Подготовлен вывод графиков амплитуд падающей волны на входе $A(0, t)=\frac{v(e)+50 * i(s)}{2}$ и выходе $A(l, t)=\frac{v(u)+50 * i(l)}{2}($ плот 1$)$, амплитуд отраженной волны на входе $B(0, t)=\frac{v^{2}(e)-50 * i(s)}{2}$ и выходе $B(l, t)=\frac{v(u)-50 * i(l)}{2}($ плот 2$)$, напряжений на входе и выходе $v(e), v(u)$ (плот 3$)$ и токов на входе и выходе $50 * i(s), 50 * i(l)$ (плот 4). 

Временной диапазон графиков выбран равным $20 \tau (\tau = 10$ нс). 


\subsection{Согласованная линия}

На схеме установим $R_{s}=R_{l}=50$ Ом, и выведем графики (через меню $Analisys/Transient/Run$). Проанализируем графики амплитуды падаюшей волны, напряжений и токов. А так же, измерив по графикам установившиеся значения $v(u)$ и $i(l) w$, убедимся в том, что источник отдает в нагрузку предельную мощность:

$P=v(u) i(l)=\frac{V^{2}}{4 R_{s}}, V=1 \mathrm{~B}:$

\[P w=v(u) i(l) w=\frac{V^{2}}{4 R_{s}} w=0,25 .\]



\subsection{Рассогласованный источник}


\subsection{Рассогласованная нагрузка}


\subsection{Рассогласованный источник и нагрузка}


\subsection{Емкостная нагрузка}


\section{Литература}

\begin{itemize}

\item Григорьев А.А. Лекции по теории сигналов. - М.: МФТИ, $2013 .$

\item Методические указания к работе №$23 (\text{Длинный цепи})$.

\end{itemize}


\end{document}

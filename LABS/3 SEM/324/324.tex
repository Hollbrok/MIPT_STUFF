\documentclass[a4paper, 12pt]{article}%тип документа

%%%Библиотеки
	%\usepackage[warn]{mathtext}	
	\usepackage[T2A]{fontenc} % кодировка
	\usepackage[utf8]{inputenc} % кодировка исходного текста
	\usepackage[english,russian]{babel} % локализация и переносы
	\usepackage{caption}
	\usepackage{listings}
	\usepackage{amsmath,amsfonts,amssymb,amsthm,mathtools}
	\usepackage{wasysym}
	\usepackage{graphicx}%Вставка картинок правильная
	\usepackage{float}%"Плавающие" картинки
	\usepackage{wrapfig}%Обтекание фигур (таблиц, картинок и прочего)
	\usepackage{fancyhdr} %загрузим пакет
	\usepackage{lscape}
	\usepackage{xcolor}
	\usepackage[normalem]{ulem}
	\usepackage{hyperref}

%%%Конец библиотек




%%%Настройка ссылок
	\hypersetup
	{
		colorlinks=true,
		linkcolor=blue,
		filecolor=magenta,
		urlcolor=blue
	}
%%%Конец настройки ссылок


%%%Настройка колонтитулы
	\pagestyle{fancy}
	\fancyhead{}
	\fancyhead[L]{Вопрос по выбору}
	\fancyhead[R]{Талашкевич Даниил, группа Б01-009}
	\fancyfoot[C]{\thepage}
%%%конец настройки колонтитулы



							\begin{document}
						%%%%Начало документа%%%%


%%%Начало титульника
\begin{titlepage}

	\newpage
	\begin{center}
		\normalsize Московский физико-технический институт \\(госудраственный 			университет)
	\end{center}

	\vspace{6em}

	\begin{center}
		\Large Лабораторная работа по электричеству\\
	\end{center}

	\vspace{1em}

	\begin{center}
		\large \textbf{Свободные колебаний в электрическом контуре [3.2.4]}
	\end{center}

	\vspace{2em}

	\begin{center}
		\large Талашкевич Даниил Александрович\\
		Группа Б01-009
	\end{center}

	\vspace{\fill}

	\begin{center}
	Долгопрудный \\2021
	\end{center}
	
\end{titlepage}
%%%Конец Титульника



%%%Настройка оглавления и нумерации страниц
	\thispagestyle{empty}
	\newpage
	\tableofcontents
	\newpage
	\setcounter{page}{1}
%%%Настройка оглавления и нумерации страниц


					%%%%%%Начало работы с текстом%%%%%%

\section{Аннотация}

\subsection{Цель работы}
\begin{enumerate}
\item   Исследование свободных колебаний в электрическом контуре.
\end{enumerate}

\subsection{В работе используются:}
\begin{itemize}
    \item Генератор импульсов
    \item электронное реле
    \item магазин сопротивлений
    \item магазин емкостей
    \item катушка индуктивности
    \item электронный осциллограф
    \item универсальный измерительный мост 

\end{itemize}

\subsection{Теоретическое вступление и модель}

В работе планируется:

\begin{enumerate}
    \item Исследовать зависимость периода свободных колебаний контура от емкости. Согласно теории, зависимость должна иметь вид (Формула Томпсона):

\begin{equation}
    T = 2\pi \sqrt{LC} \quad
\end{equation}
    

    где $T$ - период колебаний, $L$ и $C$ - индуктивность и емкость контура соответственно.

    Период планируется измерять с помощью осциллографа.

    \item Исследовать зависимость логарифмического декремента затухания от сопротивления. \\ Расчет логарифмического декремента затухания будет производиться по следующей формуле:

\begin{equation}
	\theta = \frac{1}{n} \ln\frac{W_k}{W_{k+n}} \quad    
\end{equation}
    

    где $W_i$ - энергия контура после i-того колебания.

    Энергию контура планируется высчитывать используя напряжение на конденсаторе, которое в свою очередь, измеряется с помощью осциллографа. \\

    
Согласно теории, логарифмический декремент затухания пропорционалени сопротивлению
\begin{equation}
	\lambda \propto R
\end{equation}
    
    \newpage

    \item Определить критическое сопротивление. Критическое сопротивление вычисляется по формуле:

\begin{equation}
	R_\text{кр} = 2\sqrt{\frac{L}{C}}
\end{equation}
   

    \item Определить добротность контура. Добротность планируется вычислить двумя способами, с последующим сравнением результатов. \\

    Первый способ - Через формулу для логарифмического декремента затухания. \\
    Второй способ - используя параметры контура R, L, C. \\[0.1cm]

    Формула для вычисления добротности через логарифмический декремент затухания:
\begin{equation}
	Q = \frac{\pi}{\lambda}
\end{equation}
    Формула для вычисления добротности с использованием параметров контура
\begin{equation}
 	Q = \frac{1}{R} \sqrt{\frac{L}{C}}
\end{equation}

\end{enumerate}

\section{Экспериментальная установка}

Схема установки представлена на рисунке 1.

\begin{center}

    \includegraphics[scale=0.6]{324_scheme.png} \\
    \textit{Рис. 1. Схема установки}

\end{center}

\section{Ход работы}

\subsection{Подготовка}

Проделав всю подготовительную работу: полная настройка осциллографа, ознакомиться с принципами его работы, сбор схемы, настройка емкостного магазина вместе с магазином сопротивлений, мы можем приступать к основной части работы:


\subsection{Измерение периодов свободных колебаний}

Установим на магазине сопротивлений $R = 0$ Ом и $C = 0,02$ мкФ. Подобрав частоту развертки получим изображение наших колебаний на осциллографе:


\begin{figure}[h!]
\begin{center}
\includegraphics[width = 0.55\textwidth]{my1.jpg}
\caption{Колебания в контуре}
\end{center}
\end{figure}

Подбираем частоту развёрстки ЭО так, чтобы расстояние $x_0$ между импульсами, поступающими с генератора, занимаело почти весь экран, получено значение $x_0 = 10$ см. 

Теперь, изменяя ёмкость в диапазоне от $0,02$ до $0,09$ мкФ проведем измерения периодов свободных колебаний по следующей формуле:

\[T = T_0x / (nx_0) \]

а результаты занесём в таблицу 2.

%Полученные результаты занесём в таблицу. Как будет видно из таблицы, теоритические данные сходятся с экспериментальными.

\subsection{Измерение критического сопротивления и декремента затухания}

Приняв $L = 200$ мГн, рассчитаем емкость, при которой частота собственных колебаний контура будет равна $\nu_0 = 5$ кГц.
\[C = \dfrac{1}{4 \pi^2 \nu_0^2 L} \approx 16 \text{нФ}\]
И для значений $L$ и $C$ рассчитаем $R_{\text{крит}}$
\[R_{crit} = 2\sqrt{\dfrac{L}{C}} \approx 7,1 \text{кОм}\]

Теперь будем увеличивать $R$ от нуля до какого-то $R_{\text{крит}} (\text{эксп})$, при этом будет наблюдаться картина затухающих колебаний на экране ЭО, при переходе колебательного режима в апериодический можно определить $R_{\text{крит}} (\text{эксп}) \approx 5,3 $ кОм.

Видно, что это значение не сходится с теоретическим. Всё потому, что $L$ имеет сильно отличное значение от предположенного. Результаты измерений с помощью $LCR$-измерителя предъявлены в таблице 1. Рассчитанное по ним $R_{\text{крит}} (\text{теор}) = 5,07$ кОм, что значительней сходится с экспериментальными показаниями.

Для этих значений $L$ и $C$ рассчитаем декремент затухания для каждого сопротивления из интервала $(0,1-0,3)R_{\text{крит}}$. Из этих данных по формуле
 
\[\theta = \frac{1}{n} \ln\frac{W_k}{W_{k+n}} \quad  \]

находим $\theta$, все полученные данные запишем все в таблицу 3. 


\begin{figure}[h]
\begin{center}
\includegraphics[width = 0.35\textwidth]{2.jpg}
\caption{Затухание колебаний, переход режима в апериодический}
\end{center}
\end{figure}

\newpage
\subsection{Свободные колебания на фазовой плоскости}
Попробуем понаблюдать одновременно осцилограммы тока и напряжения свободных затухающих колебаний. После проделывания всех инструкция по настройки осциллографа для этой задачи получаем следующую картину:


\begin{figure}[h!]
\begin{center}
\includegraphics[width = 0.6\textwidth]{my2.jpg}
\caption{Осцилограмма тока и напряжения свободных затухающих колебаний}
\end{center}
\end{figure}


Которая идентична той, с которой её предлагают сравнить. Рассмотрим свободные колебания на фазовой плоскости, для этого подключим место соединения катушки индуктивности и магазина сопротивлений к выходу $X$ и включим на осциллографе канал $X-Y$. В итоге мы получаем следующую картинку на экране ЭО:

\begin{figure}[h!]
\begin{center}
\includegraphics[width = 0.6\textwidth]{my3.jpg}
\caption{Фазовая диаграмма для свободных колебаний}
\end{center}
\end{figure}

Рисунок совпадает с теоретическим (рис 2.2б в описании к работе).



\subsection{Добротность свободных колебаний в контуре}
Добротность можно найти по формуле 
\[Q = \dfrac{\pi}{\theta}\]
Найдем ее для $R_{max} = 2,25$ кОм и для $R_{min} = 0,75$ кОм из графика и фазовой диаграммы. Итоговые результаты запишем в таблицу.

Так же добротность можно найти и из теоретических соображений по формуле
\[Q = \dfrac{1}{R}\sqrt{\dfrac{L}{C}}\]

Результаты так же занесем в таблицу, и в итоге мы получаем таблицу 5 со всеми результатами данного эксперимента, по которой мы можем сравнить все способы получения и их погрешности.

\section{Обработка результатов}

\begin{itemize}

\item теоретические и экспериментальные значение $T$, при различных значения $C$, а так же погрешность полученных значений занесены в таблицу 2.

\item По результатам из таблицы 3 получим таблицу значений для $\frac{1}{\theta^2}$, $\frac{1}{R_{\sum}^2}$. По этим величинам построим график зависимости $1 / \Theta^{2}=f\left[1 /\left(R_{\Sigma}^{2}\right)\right] .$ Приняв обозначения $1 / \Theta^{2}=Y, \quad 1 /\left(R_{\Sigma}^{2}\right)=X$, и покажем, что $R_{\mathrm{kp}}=2 \pi \sqrt{\Delta Y / \Delta X} \quad$.

\item Полученный результаты логарифмического декремента затухания занесены в таблицу 3.

\item Теоретическое значение $R_{\text{крит}}$,а так же сравнение его с полученными экспериментальными данными было сделано ранее. 

\item $Q(\theta_{max}) = \frac{\pi}{2,9} = 1,08$ ; $Q(\theta_{min}) = \frac{\pi}{0,8} = 3,93$. Теперь рассчитаем $Q$ через параметры контура: 

\[Q = \frac{1}{R} \sqrt{\frac{L}{C}} \Rightarrow  Q(\theta_{max})(\text{теор}) = 1, 12; Q(\theta_{min})(\text{теор}) = 3,37\] 

\end{itemize}

Все погрешности косвенных измерений в данной работе были рассчитаны по следующим формулам:

$$ z=f(a, b, c, \ldots) $$

$$ \Delta_{z}=\sqrt{\left(\frac{\partial f}{\partial a} \Delta_{a}\right)^{2}+\left(\frac{\partial f}{\partial b} \Delta_{b}\right)^{2}+\left(\frac{\partial f}{\partial c} \Delta_{c}\right)^{2}+\ldots} $$

где $\frac{\partial f}{\partial a}, \frac{\partial f}{\partial b}, \frac{\partial f}{\partial c}, \ldots-$ частные производные искомой функции $z$.

Погрешности прямых измерений рассчитывались как погрешность приборов.

Погрешность коэффициентов $k,b: y = kx + b$ определялись по МНК.

\section{Графики и таблицы}

Зависимости показаний $LCR$-метра $L, R_L$ от частоты:   

\begin{table}[h!]
\begin{center}
\begin{tabular}{|c|c|c|}
\hline
$\nu$, Гц & $L$, мГн & $R_L$, Ом \\ \hline
50        & 148,23    & 9,77      \\ \hline
1000      & 142,27    & 12,45      \\ \hline
5000      & 143,16    & 17,50      \\ \hline
\end{tabular}
\caption{Некоторые параметры катушки индуктивности}
\end{center}
\end{table}

\begin{table}[h!]
\begin{center}
\begin{tabular}{|c|c|c|c|c|c|c|c|c|}
\hline
С, нФ & $N$ периодов & $x_0$, (дел.) & $x$, (дел.) & $T_{\text{теор}}$, мс & $T_{\text{эксп}}$, мс & $\sigma_T$, мс \\ \hline
20               & 29  & 10  &  9,7    & 0,33           & 0,31            & 0,03           \\ \hline
40               & 2   & 10  &  1,0    & 0,50           & 0,49            & 0,03           \\ \hline
80               & 3   & 10  &  2,0    & 0,67           & 0,66            & 0,03           \\ \hline
120              & 6   & 10  &  5,0    & 0,83           & 0,81            & 0,03           \\ \hline
200              & 4   & 10  &  4,2    & 1,05           & 1,04            & 0,03           \\ \hline
300              & 3   & 10  &  4,0    & 1,33           & 1,31            & 0,04           \\ \hline
400              & 2   & 10  &  3,0    & 1,50           & 1,50            & 0,04           \\ \hline
500              & 2   & 10  &  3,5    & 1,75           & 1,72            & 0,04           \\ \hline
\end{tabular}
\caption{Таблица данных измерения периода свободных колебаний и сравнение с теорией}
\end{center}
\end{table}

Из этой таблицы строим 2 графика: $T(\sqrt{C})$ и $T(\text{эксп}) = f\left[T(\text{теор})\right]$ :

\begin{figure}[h!]
\begin{center}
\includegraphics[width = 1.0\textwidth]{graph11.jpg}
\caption{График зависимости $T(\sqrt{C})$}
\end{center}
\end{figure}

\begin{figure}[h!]
\begin{center}
\includegraphics[width = 1.0\textwidth]{graph2.jpg}
\caption{График зависимости $T_{\text{эксп}}(T_{\text{теор}})$}
\end{center}
\end{figure}

\begin{figure}[h!]
\begin{center}
\includegraphics[width = 0.9\textwidth]{graph3.jpg}
\caption{График зависимости $T_{\text{эксп}}(T_{\text{теор}})$}
\end{center}
\end{figure}

\begin{table}[h!]
\begin{center}
\begin{tabular}{|c|c|c|c|c|c|}
\hline
$R$, Ом & $U_1$, дел & $U_2$, дел & n & $\theta$ & $\sigma_{\theta}$ \\ \hline
750  & 2,2  & 0,2        & 3 & 0,8    & 0,08              \\ \hline
1500 & 1    & 0,2        & 1 & 1,6    & 0,3              \\ \hline
2250 & 3,8    & 0,15     & 1 & 2,9    & 0,5              \\ \hline
\end{tabular}
\caption{Декремент затухания для фазовой диаграммы}
\end{center}
\end{table}

\begin{table}[h!]
\begin{center}
\begin{tabular}{|c|c|c|c|}
\hline
\textbf{$R_{\sum}$} & \textbf{$\theta$} & \textbf{$Y$} & \textbf{$X$} \\ \hline
767,5                 & 0,8      & 1,56     & 1,70E-06 \\ \hline
1517,5                & 1,6      & 0,39     & 4,34E-07 \\ \hline
2267,5                & 2,9      & 0,12     & 1,94E-07 \\ \hline
\end{tabular}
\caption{Графический метод определения $R_{\text{кр}}$}
\end{center}
\end{table}

\newpage 

\begin{table}[h!]
\begin{tabular}{|c||c|c|c||c|c|c|c|}
\hline \multicolumn{1}{|l||}{} & \multicolumn{3}{c||}{$R_{\text {кр }}$}  &  & \multicolumn{3}{|c|}{$Q$} \\
\cline { 2 - 4 } \cline { 6 - 8 }$L_{\text {кат }}$ & Teop. & Подбор & Граф. & $R$ & Teop. & $f(\Theta)$ & Спираль \\
\hline \hline & & & & $\max(2,25)$  & 1,12 $\pm 0,11$ & 1,08 $\pm 0,06$ & 1,04 $\pm 0,25$\\
0,143 & 5070 & 5300 $\pm 230$ & 5460 $\pm 500$ & $\min(0,75)$ & 3,37 $\pm 0,3$ & 3,93 $\pm 0,22$ & 3,4 $\pm 0,78$\\
\hline
\end{tabular}
\caption{результаты методов определения $R_{\text{кр}}$ и $Q$}
\end{table}

\newpage

\section{Вывод}

Как видно из таблицы 5, наилучший способ измерения добротности --- с помощью графика, потому что получаются наиболее близкие значения с меньшими погрешностями. Так же была получена линейная зависимость $\theta(R)$, что согласуется с теорией.

\section{Литература}

\begin{enumerate}
\item \textbf{Лабораторный практикум по общей физике:} Учебное пособие. В трех томах. Т. 2. Электричество и магнетизм /Гладун А.Д., Александров Д.А., Берулёва Н.С. и др.; Под ред. А.Д. Гладуна - М.: МФТИ, 2007. - 280 с.
\item \textbf{Дополнительное описание лабораторной работы 3.2.4}: Свободные колебания в электрическом контуре; Под ред. МФТИ, 2018 г. - 4 с.
\end{enumerate}

					\end{document}
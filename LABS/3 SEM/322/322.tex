\documentclass[a4paper, 12pt]{article}%тип документа

%%%Библиотеки
	%\usepackage[warn]{mathtext}	
	\usepackage[T2A]{fontenc} % кодировка
	\usepackage[utf8]{inputenc} % кодировка исходного текста
	\usepackage[english,russian]{babel} % локализация и переносы
	\usepackage{caption}
	\usepackage{listings}
	\usepackage{amsmath,amsfonts,amssymb,amsthm,mathtools}
	\usepackage{wasysym}
	\usepackage{graphicx}%Вставка картинок правильная
	\usepackage{float}%"Плавающие" картинки
	\usepackage{wrapfig}%Обтекание фигур (таблиц, картинок и прочего)
	\usepackage{fancyhdr} %загрузим пакет
	\usepackage{lscape}
	\usepackage{xcolor}
	\usepackage[normalem]{ulem}
	\usepackage{hyperref}

%%%Конец библиотек




%%%Настройка ссылок
	\hypersetup
	{
		colorlinks=true,
		linkcolor=blue,
		filecolor=magenta,
		urlcolor=blue
	}
%%%Конец настройки ссылок


%%%Настройка колонтитулы
	\pagestyle{fancy}
	\fancyhead{}
	\fancyhead[L]{Лабораторная работа}
	\fancyhead[R]{Талашкевич Даниил, группа Б01-009}
	\fancyfoot[C]{\thepage}
%%%конец настройки колонтитулы



							\begin{document}
						%%%%Начало документа%%%%


%%%Начало титульника
\begin{titlepage}

	\newpage
	\begin{center}
		\normalsize Московский физико-технический институт \\(госудраственный 			университет)
	\end{center}

	\vspace{6em}

	\begin{center}
		\Large Лабораторная работа по электричеству\\
	\end{center}

	\vspace{1em}

	\begin{center}
		\large \textbf{Резонанс напряжений в последовательном контуре [3.2.2]}
	\end{center}

	\vspace{2em}

	\begin{center}
		\large Талашкевич Даниил Александрович\\
		Группа Б01-009
	\end{center}

	\vspace{\fill}

	\begin{center}
	Долгопрудный \\2021
	\end{center}
	
\end{titlepage}
%%%Конец Титульника



%%%Настройка оглавления и нумерации страниц
	\thispagestyle{empty}
	\newpage
	\tableofcontents
	\newpage
	\setcounter{page}{1}
%%%Настройка оглавления и нумерации страниц


					%%%%%%Начало работы с текстом%%%%%%
		
\section{Аннотация}
\textbf{Цель работы:} исследование резонанса напряжений в последовательном
колебательном контуре с изменяемой ёмкостью, получение амплитудно­
частотных и фазово-частотных характеристик, определение основных па­
раметров контура.\\
\textbf{В работе используются:} генератор сигналов, источник напряжения,
нагрузкой которого является последовательный колебательный контур с
переменной ёмкостью, двухканальный осциллограф, цифровые вольтмет­
ры.



\subsection{Теоретическое вступление и модель}

XXX

\subsection{Экспериментальная установка}

В данной работе изучаются резонансные явления в последовательном колебательном контуре (резонанс напряжений). Схема экспериментального стенда показана на рис. $1 .$ Синусоидальный сигнал от генератора поступает на вход управляемого напряжсением источника напрялсения (см., например, [3]), собранного на операционном усилителе, питание которого осуществляется встроенным блоком-выпрямителем от сети $\sim 220 \mathrm{~B}$ (цепь питания на схеме не показана). Источник напряжсения (источник с нулевым внутренним сопротивлением) обеспечивает с высокой точностью постоянство амплитуды сигнала $\mathcal{E}=\mathcal{E}_{0} \cos \left(\omega t+\varphi_{0}\right)$ на меняющейся по величине нагрузке - последовательном колебательном контуре, изображённом на рис. 1 в виде эквивалентной схемы.

Источник напряжения, колебательный контур и блок питания заключены в отдельный корпус, отмеченный на рисунке штриховой линией. На корпусе имеются коаксиальные разъёмы «Вход», «$U_1$» и «$U_{2}$», а также переключатель магазина ёмкостей $C_{n}$ с указателем номера $n=1,2, \ldots$ 7. Величины ёмкостей $C_{n}$ указаны на установке. Напряжение $\mathcal{E}$ на контуре через разъём «$U_1$» попадает одновременно на канал 1 осциллографа и вход 1-го цифрового вольтметра. Напряжение на конденсаторе $U_{C}$ подаётся через разъём «$U_2$» одновременно на канал 2 осциллографа и вход 2-го цифрового вольтметра.

\begin{center}

    \includegraphics[scale=0.65]{pics/scheme1.png} \\
   % \textit{Рис. 1. Схема установки}

\end{center}

\section{Ход работы}

\subsection{Подготовка}


XXX\\
SOME MORE SUBSECTIONS

\section{Обработка результатов}

\begin{itemize}

\item XXX

\item XXX

\item XXX

\item XXX

\item XXX

\end{itemize}

XXX

\section{Графики и таблицы}

XXX

\section{Вывод}

XXX

\section{Литература}

\begin{enumerate}
\item \textbf{Лабораторный практикум по общей физике:} Учебное пособие. В трех томах. Т. 2. Электричество и магнетизм /Гладун А.Д., Александров Д.А., Берулёва Н.С. и др.; Под ред. А.Д. Гладуна - М.: МФТИ, 2007. - 280 с.
\end{enumerate}		
		
					
\end{document}
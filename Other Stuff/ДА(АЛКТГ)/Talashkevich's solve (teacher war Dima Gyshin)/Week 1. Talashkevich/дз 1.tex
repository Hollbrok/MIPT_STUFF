\documentclass[a4paper,12pt]{article} % тип документа


%  Русский язык
\usepackage[T2A]{fontenc}			% кодировка
\usepackage[utf8]{inputenc}			% кодировка исходного текста
\usepackage[english,russian]{babel}	% локализация и переносы


% Математика
\usepackage{amsmath,amsfonts,amssymb,amsthm,mathtools}


\usepackage{wasysym}

%Заговолок
\author{Талашкевич Даниил Александрович}

\title{Домашнее задание по дискретному анализу. Неделя 1. Алгебра логики. Введение}

\date{\today}


\begin{document}
\maketitle

\newpage
\begin{center}
\section*{Задача 1}
\end{center}

Согласно условию задачи,
\[ \neg{(x=y)} \land((y<x) \to (2z>x)) \land((x<y) \to (x>2z)) =1\]
Так как это выражение - истина, тогда истине равны:

$a)$ $\neg{(x=y)}=1$ 

$b)$ $(y<x) \to (2z>x)=1$

$c)$ $(x<y) \to (x>2z)=1$ \\

Из пункта а) следует, что $x \neq y$, то есть $x \neq 16$.

Пункт б) выполняется всегда, кроме случая:

\[\begin{cases}
  (y < x) = 1\\
  (2z>x)=0
\end{cases}\] 
\[\begin{cases}
  x>y\\
  x\geqslant 2z
\end{cases}\]
 \[\begin{cases}
  x>16\\
  x\geqslant 14
\end{cases}\]
\[ x>16\]
То есть пункт б) выполняется, если
\[ x\leqslant15\]
Пункт в) отличается от б) только знаками неравенств:
\[ x\geqslant15\]
В итоге получили систему уравнений:
\[\begin{cases}
  x\geqslant15\\
  x\leqslant15\\
  x \neq 16
\end{cases}\]
\[x = 15\]
\begin{flushright}
\begin{large}
\textbf {Ответ: 15}
\end{large}
\end{flushright}
\newpage
\begin{center}
\section*{Задача 2}
\end{center}
\[f(x,y,z)=\neg{((x \wedge \neg y)\wedge z)}\]
Найдём все значения функции для построения таблицы истинности:
\[f(0,0,0)=\neg{((0 \wedge \neg 0)\wedge 0)}=\neg{(0\wedge 0)}=1\]
\[f(0,0,1)=\neg{((0 \wedge \neg 0)\wedge 1)}=\neg{(0\wedge 1)}=1\]
\[f(0,1,0)=\neg{((0 \wedge \neg 1)\wedge 0)}=\neg{(0\wedge 0)}=1\]
\[f(0,1,1)=\neg{((0 \wedge \neg 1)\wedge 1)}=\neg{(0\wedge 1)}=1\]
\[f(1,0,0)=\neg{((1 \wedge \neg 0)\wedge 0)}=\neg{(1\wedge 0)}=1\]
\[f(1,0,1)=\neg{((1 \wedge \neg 0)\wedge 1)}=\neg{(1\wedge 1)}=0\]
\[f(1,1,0)=\neg{((1 \wedge \neg 1)\wedge 0)}=\neg{(0\wedge 0)}=1\]
\[f(1,1,1)=\neg{((1 \wedge \neg 1)\wedge 1)}=\neg{(1\wedge 0)}=1\]
\begin{center}
\begin{tabular}{|c|c|c|c|}
\hline
$x$ & $y$ & $z$ & $f(x,y,z)$ \\
\hline
0 & 0 & 0 & 1 \\
\hline
0 & 0 & 1 & 1 \\
\hline
0 & 1 & 0 & 1 \\
\hline
0 & 1 & 1 & 1 \\
\hline
1 & 0 & 0 & 1 \\
\hline
1 & 0 & 1 & 0 \\
\hline
1 & 1 & 0 & 1 \\
\hline
1 & 1 & 1 & 1 \\
\hline
\end{tabular}
\end{center}

\newpage
\begin{center}
\section*{Задача 3}
\end{center}
\[1\oplus x_1 \oplus x_2 = (x_1\rightarrow x_2) \wedge (x_2\rightarrow x_1) \]
Для доказательства рассмотрим, когда (в каких случаях) оба выражения равны истине:\\

$1)$ $f_1(x_1, x_2)= (1\oplus x_1) \oplus x_2 =1$

$ a)$ если $x_1=0,$ то $x_2=0,$

$ b)$ если $x_1=1,$ то $x_2=1,$
То есть $x_1 = x_2,$ если $f_1(x_1, x_2)=1$ \\

$2)$ $f_2(x_1, x_2)= (x_1\rightarrow x_2) \wedge (x_2\rightarrow x_1)=1$

Заметим, что $x_1 = x_2 =0$ или $x_1 = x_2 =1$, иначе возникает ситуация $1\rightarrow 0 = 0$, то есть $x_1 = x_2,$ если $f_2(x_1, x_2)=1$ \\

Оба выражения равны истине только тогда, когда $x_1 = x_2$, в остальных случаях $(x_1 \neq x_2)$ они равны нулю (лжи), то есть булевы функции $f_1(x_1, x_2)$ и $f_2(x_1, x_2)$ ведут себя одинаково при различных $x_1$ и $x_2$, значит они эквивалентны.


\begin{flushright}
\begin{large}
\textbf {Доказано}
\end{large}
\end{flushright}

\begin{center}
\section*{Задача 4}
\end{center}


$a)$ $x\wedge (y\rightarrow z) =(x\wedge y)\rightarrow(x\wedge z)$\\

Пусть $x=0$, тогда выражение слева в $a)$ всегда равно нулю \[(0\wedge (y\rightarrow z))=0\]

Значит \[(x\wedge y)\rightarrow(x\wedge z)=0\]

\[\begin{cases}
  x\wedge y = 1\\
  x\wedge z = 0
\end{cases}\]
\[x=y=1\]
Получили, что $x=1$, предполагая, что $x=0$.
 \underline{Противоречие. Дистрибутивность не выполянется.}\\
 $b)$ $x\oplus (y \leftrightarrow z) = (x\oplus y)\leftrightarrow (x\oplus z)$ \\

Предположим, что выражение слева $(x\oplus (y \leftrightarrow z))$ равно истине, тогда:
\begin{equation}x\oplus (y \leftrightarrow z)=1
\end{equation}
\begin{equation}(x\oplus y)\leftrightarrow (x\oplus z)=1
\end{equation}

Рассмотрим решение уравнения $(1)$: случай 1:
\[\begin{cases}
  x\oplus y = 1\\
  x\oplus z = 1
\end{cases}\]
\[y=z\neq x\]
Рассмотрим случай 2:
\[\begin{cases}
  x\oplus y = 0\\
  x\oplus z = 0
\end{cases}\]
\[y=z = x\]
В любом случае в решении $(x\oplus y)\leftrightarrow (x\oplus z)=1$ выполняется $y=z$.
  
Решением уравнения $(2)$ являются 2 системы:
 \begin{equation}\begin{cases}
  x = 0\\
  y =z
\end{cases} \end{equation}
 \begin{equation}\begin{cases}
  x = 1\\
  y \neq z
\end{cases} \end{equation}
 То есть одно из решений содержит $y \neq z$, но в решении уравнения $(1)$ всегда $y=z$.
\underline{Противоречие. Дистрибутивность не выполянется.}
\newpage
\begin{center}
\section*{Задача 5}
\end{center}

$a)$ \underline{Коммутативность для импликации} $x\rightarrow y =y\rightarrow x$ \underline{не выполняется}, так как если $x=0, y=1$, то $0\rightarrow 1=1$, но $1\rightarrow 0=0$ \\

$b)$ \underline{Ассоциативность} $(x\rightarrow y)\rightarrow z = x\rightarrow (y\rightarrow z) $ \underline{не выполняется}, так как при если $x=0$, то $x\rightarrow (y\rightarrow z)=1$ при любых $y$ и $z$, при этом $x\rightarrow y=1$, но если $z=0$, то $(x\rightarrow y)\rightarrow z =0$, в то время как $x\rightarrow (y\rightarrow z)=1$ \\

\begin{center}
\item \section*{Задача 6}
\end{center}

$a)$ $f(x_1,x_2,x_3)=00111100$

Для наглядности составим  таблицу истинности:
\begin{center}
\begin{tabular}{|c|c|c|c|}
\hline
$x_1$ & $x_2$ & $x_3$ & $f(x_1,x_2,x_3)$ \\
\hline
0 & 0 & 0 & 0 \\
\hline
0 & 0 & 1 & 0 \\
\hline
0 & 1 & 0 & 1 \\
\hline
0 & 1 & 1 & 1 \\
\hline
1 & 0 & 0 & 1 \\
\hline
1 & 0 & 1 & 1 \\
\hline
1 & 1 & 0 & 0 \\
\hline
1 & 1 & 1 & 0 \\
\hline
\end{tabular}
\end{center}

Заметим, что при $x_1=x_2$ выходит $f(x_1,x_2,x_3)=0$, а в других случаях $f(x_1,x_2,x_3)=1$, значит \underline{$x_1$ и $x_2$ - существенные переменные, а $x_3$ - фиктивная.} \\

$b)$ $g(x_1,x_2,x_3)=(x_1\rightarrow (x_1\vee x_2))\rightarrow x_3$

Если $x_1=1$, то $(x_1\rightarrow (x_1\vee x_2))=(1\rightarrow 1)=1$.

Если $x_1=0$, то $(x_1\rightarrow (x_1\vee x_2))=(0\rightarrow (0\vee x_2))=1$.

Получили, что $(x_1\rightarrow (x_1\vee x_2))=1$ не зависит от $x_1$ и тем более от $x_2$, значит $g(x_1,x_2,x_3)=1\rightarrow x_3$.
Итак, \underline{$x_1$ и $x_2$ - фиктивные переменные, а $x_3$ - существенная переменная.}

\newpage
\begin{center}
\section*{Задача 7}
\end{center}
\[f(x_1,...,x_n)=(x_1\vee f(0,x_2,...,x_n))\wedge (\neg x_1 \vee f(1,x_2,...,x_n))\]

$1)$ Пусть $f(0,x_2,...,x_n)=0$, тогда:
\[f(0,x_2,...,x_n)=(0\vee f(0,x_2,...,x_n))\wedge (1 \vee f(1,x_2,...,x_n))=0\wedge 1 =0\]
Выражение выполняется.

$2)$ Пусть $f(0,x_2,...,x_n)=1$, тогда:
\[f(0,x_2,...,x_n)=(0\vee f(0,x_2,...,x_n))\wedge (1 \vee f(1,x_2,...,x_n))=1\wedge 1 =1\]
Выражение выполняется.

$3)$ Пусть $f(1,x_2,...,x_n)=0$, тогда:
\[f(1,x_2,...,x_n)=(1\vee f(0,x_2,...,x_n))\wedge (0 \vee f(1,x_2,...,x_n))=1\wedge 0 =0\]
Выражение выполняется.

$4)$ Пусть $f(1,x_2,...,x_n)=1$, тогда:
\[f(0,x_2,...,x_n)=(1\vee f(0,x_2,...,x_n))\wedge (0 \vee f(1,x_2,...,x_n))=1\wedge 1 =1\]

Выражение выполняется.

Итак, исходное равенство выполняется для любых функций с любыми значениями $x_1$. Интересно также отметить, что равенство выполняется для любого аргумента $x_i$ $(0\leqslant i\leqslant n)$, так как аргументы являются независимыми.

\newpage
\begin{center}
\section*{Задача 8}
\end{center}
\[x_1^{\alpha_1}\wedge x_2^{\alpha_2} \wedge ... \wedge x_n^{\alpha_n}=1\]

Равенство возможно только в одном случае: $x_i^{\alpha_i}=1$ $(1\leqslant i\leqslant n)$. Значение $x_i^{\alpha_i}$ зависит от $x_i$ и $\alpha_i$: если $x_i=0$, то $\alpha_i=0$, чтобы $x_i^{\alpha_i}=1$ и если $x_i=1$, то $\alpha_i=1$, чтобы $x_i^{\alpha_i}=1$. Определённому значению $x_i$ соответсвует определённое $\alpha_i$, значит если выбран определённый набор $x_1,...,x_n$, ему будет соответствовать единственный набор $\alpha_1,...,\alpha_n$ \\\\
\begin{flushright}
\begin{large}
\textbf {Доказано}
\end{large}
\end{flushright}

\begin{center}
\section*{Задача 9}
\end{center}
\[\bigvee \limits_{i, j} (x_i\oplus x_j)=(x_1\vee x_2 \vee ... \vee x_n)\wedge (\overline{x_1} \vee \overline{x_2} \vee ... \vee \overline{x_n}) \]
$1)$ $\bigvee \limits_{i, j} (x_i\oplus x_j)=1$, если хотя бы одна комбинация $x_i$ и $x_j$ отличается по значениям. В этом же случае $(x_1\vee x_2 \vee ... \vee x_n)=1$, так как среди $x_i$ будет по крайней мере одна единица, и $(\overline{x_1} \vee \overline{x_2} \vee ... \vee \overline{x_n})= 1$, так как по крайней мере найдётся одно значение $x_i=0$, а значит $\overline{x_i}=1$. Значит
\[\bigvee \limits_{i, j} (x_i\oplus x_j)=(x_1\vee x_2 \vee ... \vee x_n)\wedge (\overline{x_1} \vee \overline{x_2} \vee ... \vee \overline{x_n})=1 \]

Получается, что равенство в условии выполняется.\\
$2)$ $\bigvee \limits_{i, j} (x_i\oplus x_j)=0$, если все значения $x_i$ равны (0 или 1), значит либо $(x_1\vee x_2 \vee ... \vee x_n)=0$, либо $(\overline{x_1} \vee \overline{x_2} \vee ... \vee \overline{x_n})= 0$, тогда
\[\bigvee \limits_{i, j} (x_i\oplus x_j)=(x_1\vee x_2 \vee ... \vee x_n)\wedge (\overline{x_1} \vee \overline{x_2} \vee ... \vee \overline{x_n})=0 \]

Равенство в условии снова выполняется, значит оно справедливо для любых значений $x_i$.

\newpage
\begin{center}
\section*{Задача 10}
\end{center}
\par
Пусть булева функция выражается только через связки $\vee$ и $\wedge$. Заметим, что эти связки могут только либо сохранять предыдущие значения выражений (переменных), либо увеличивать их до 1, значит функции, использующие только эти связки - нестрого возрастающие. Получается, что нестрого или строго убывающую функцию эти связки описать не могут (перевести 1 в 0), для этого как минимум требуется использовать связку $\neg$.
Значит существует убывающая функция, которая не может быть описана только связками $\vee$ и $\wedge$.


\end{document}


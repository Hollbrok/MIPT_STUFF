
\documentclass[a4paper,12pt]{article} % тип документа


% Русский язык
\usepackage[T2A]{fontenc} % кодировка
\usepackage[utf8]{inputenc} % кодировка исходного текста
\usepackage[english,russian]{babel} % локализация и переносы


% Математика
\usepackage{amsmath,amsfonts,amssymb,amsthm,mathtools}


\usepackage{wasysym}

%Заговолок
\author{Талашкевич Даниил Александрович}

\title{Домашнее задание по дискретному анализу. Неделя 2. Множества и логика}

\date{\today}

\begin{document}

\maketitle
\thispagestyle{empty}

\newpage
\setcounter{page}{1}
\begin{center}
\itshape
\bfseries
{ \Large Problems:}
\end{center}
{\bf 1.} Верно ли, что для любых множеств A и B выполняется равенство:

$(A\setminus B)\cap ((A\cup B)\setminus (A\cap B)) = A\setminus B ? $
\begin{center}
\bfseries
{\Large Решение: }
\end{center}
Рассмотрим левую часть: 

$(A\setminus B)\cap ((A\cup B)\setminus (A\cap B))  = (A\cap \overline{B})\cap ((A\cup {B}) \cap (\overline{A} \cup \overline{B})) =$

$ = (A\cap \overline{B}) \cap (((A \cap \overline{A}) \cup (\overline{A}\cap B)) \cup ((A \cap \overline{B}) \cup (B \cap \overline{B}))) = (A \cap \overline{B}) \cap$

$\cap ((A \cap \overline{B}) \cup (\overline{A}\cap B)) = A \cap \overline{B} = A\setminus B.$\\
Что и требовалось доказать.

\begin{flushright}
\begin{large}
\textbf {Ответ: верно}
\end{large}
\end{flushright}

{\bf 2.} Верно ли, что для любых множеств A, B и C выполняется равенство: $((A \setminus B) \cup (A \setminus C)) \cap (A \setminus (B \cup C)) = A \setminus (B \cup C) $ ?

\begin{center}
\bfseries
{\Large Решение: }
\end{center}
Используя законы де Моргана и законы ассоциативности, коммутативности, импликации, получим:

$((A \cap \overline{B} )\cup (A \cap \overline{C})) \cap (A \cap ( \overline{B \cap C} )) = (A \cap (\overline{B} \cup \overline{C})) \cap (( A \cap (\overline{B} \cup \overline{C})) =$

$= A \cap ( \overline{B} \cup \overline{C}) = A \cap (\overline{B \cap C}) = A \setminus ( B \cap C) \neq A\setminus (B \cup C).$ 
\begin{flushright}
\begin{large}
\textbf {Ответ: Не верно для любых множеств A, B и C (верно только при равенстве множеств B и C).}
\end{large}
\end{flushright}

       
{\bf 3.} Верно ли, что для любых множеств A, B и C выполняется равенство: $(A \cap B) \setminus C = (A \setminus C) \cap (B \setminus C)?$
\begin{center}
\bfseries
{\Large Решение: }
\end{center}
Используя законы де Моргана и законы ассоциативности, коммутативности, импликации, получим:

$(A\cap B)\cap \overline{C} = (A \cap \overline{C}) \cap (B \cap \overline{C}) = (A \setminus C) \cap (B\setminus C).$
 
\begin{flushright}
\begin{large}
\textbf {Ответ: верно.}
\end{large}
\end{flushright}

\newpage
{\bf 4.} Верно ли, что для любых множеств A и B выполняется включение $(A \cup B) \setminus (A \setminus B) \subseteq B$?
\begin{center}
\bfseries
{\Large Решение: }
\end{center}
Используя законы де Моргана и законы ассоциативности, коммутативности, импликации, получим:

$(A\cup B) \cap (\overline{A\cap \overline{B}}) = (A\cup B) \cap (\overline{A} \cup B) = B \cup (A \cap \overline{A}) = B.$

$B \subseteq B$ - истина. 

\begin{flushright}
\begin{large}
\textbf {Ответ: верно.}
\end{large}
\end{flushright}

{\bf 5.} Пусть P = [10, 40]; Q = [20, 30]; известно, что отрезок A удовлетворяет соотношению

\[((x \in A) \rightarrow (x \in P)) \wedge ((x \in Q) \rightarrow (x \in A)).\]
\begin{center}
\bfseries
{\Large Решение: }
\end{center}
Так как искомое выражение истино, то истины оба операнда:$$(x \in A)\rightarrow(x \in P).$$  
$$(x \in Q) \rightarrow (x \in A).$$ Тогда первое принимает $0$ только если $1 \rightarrow 0$ Значит случай $\overline{A} = 1 $ и $P = 0$ не реализуется $\Rightarrow A\subseteq P.$\\
Второе утверждение: $(x \in Q) \rightarrow (x \in A)$. Аналогичные рассуждения приводят к тому, что $Q \subseteq A.$\\
Значит A начинается на промежутке [10;20], а заканчивается на промежутке [30;40] $\Rightarrow \operatorname{Lenght}{A_{min}} = 10$, а $ \operatorname{Lenght}{A_{max}} = 30.$

\begin{flushright}
\begin{large}
\textbf {Ответ: минимальная -- 10, максимальная -- 30.}
\end{large}
\end{flushright}

{\bf 6.} Про множества A, B, X, Y известно, что $A \cap X = B \cap X, A \cup Y = B \cup Y .$ Верно ли, что тогда выполняется равенство $A \cup (Y \setminus X) = B \cup (Y \setminus X)$?
\begin{center}
\bfseries
{\Large Решение: }
\end{center}
Чтобы доказать равенство множеств нужно доказать равенство дополнений:

$\overline{A\cup (Y \setminus X)}= \overline{A}\cap (\overline{Y\setminus X}) = \overline{A}\cap (\overline{Y\cap\overline{X}}) = \overline{A}\cap (\overline{Y}\cup X) =$

$=(\overline{A} \cap \overline{Y}) \cup (\overline{A} \cap X) = (\overline{A \cup Y}) \cup (X \setminus (X \cap A)).$\\
Аналогично для правой части равенства $A \cup (Y \setminus X) = B \cup (Y \setminus X)$:

$B \cup (Y \setminus X) = (\overline{B \cup Y}) \cup (X \setminus (X \cap B)).$\\
Исходя из условия: $A \cap X = B \cap X, A \cup Y = B \cup Y .$ 
Получаем:

$ (\overline{A \cup Y}) \cup (X \setminus (X \cap A)) = (\overline{B \cup Y}) \cup (X \setminus (X \cap B)).$
	

\begin{flushright}
\begin{large}
\textbf {Ответ: верно.}
\end{large}
\end{flushright}


{\bf 7.} Пусть $A_{1} \supseteq A_{2} \supseteq A_{3} \supseteq \ldots \supseteq A_{n} \supseteq \ldots $ — невозрастающая последовательность множеств.
Известно, что $A_{1} \setminus A_{4} = A_{6} \setminus A_{9}$. Докажите, что $A_{2} \setminus A_{7} = A_{3} \setminus A_{8}$.

\begin{center}
\bfseries
{\Large Решение: }
\end{center}
\begin{proof}
Введем новые множества X такие, что:

\begin{align}
A_{1}\setminus A_{2} = A_{1}\cap \overline{A_{2}} = Y_{1};\\
A_{2}\setminus A_{3} = A_{2}\cap \overline{A_{3}} = Y_{2};\\
\dots\\
A_{n}\setminus A_{n+1} = A_{n}\cap \overline{A_{n+1}} = Y_{n}.
\end{align}

Тогда преобразуем условие до вида:
\begin{align}
\hspace{9mm} A_{1}\setminus A_{4} = Y_{1} \cup Y_{2} \cup Y_{3};\\
\hspace{9mm} A_{6}\setminus A_{9} = Y_{6} \cup Y_{7} \cup Y_{8}.
\end{align}
По условию $A_{1}\setminus A_{4} = A_{6}\setminus A_{9}$ . Тогда получается, что:
\begin{align}
\hspace{11mm} Y_{1}, Y_{2}, Y_{3} ,Y_{6}, Y_{7} ,Y_{8} \subseteq  \emptyset.
\end{align}
Найдем, чему будут равны $A_{2} \setminus A_{7} $ и $ A_{3} \setminus A_{8}$:

$A_{2} \setminus A_{7} = Y_{2}\cup Y_{3}\cup Y_{4}\cup Y_{5}\cup Y_{6} = Y_{4}\cup Y_{5}$ , т. к. ($ Y_{1}, Y_{2}, Y_{3} ,Y_{6}, Y_{7} ,Y_{8} \subseteq  \emptyset.$)

$ A_{3} \setminus A_{8} = Y_{3}\cup Y_{4}\cup Y_{5}\cup Y_{6}\cup Y_{7} = Y_{4}\cup Y_{5}$ , т. к. ($ Y_{1}, Y_{2}, Y_{3} ,Y_{6}, Y_{7} ,Y_{8} \subseteq  \emptyset.$)\\
Следовательно $A_{2} \setminus A_{7} $ и $ A_{3} \setminus A_{8}$, что и требовалось доказать.
\end{proof}

\begin{flushright}
\begin{large}
\textbf {Ответ: доказано.}
\end{large}
\end{flushright}


{\bf 8.} Пусть A, B, C, D — такие отрезки прямой, что $A \bigtriangleup B = C \bigtriangleup D$ (симметрические разности равны). Верно ли, что выполняется включение $A \cap B \subseteq C$?
\newpage
\begin{center}
\bfseries
{\Large Решение: }
\end{center}
Без ограничений общности положим, что $A \subseteq B$ и $C \subseteq D$, тогда:

$A\bigtriangleup B = (A \cup B) \setminus (A \cap B) = B\setminus A.$

$C\bigtriangleup D = (C \cup D) \setminus (C \cap D) = D\setminus C.$\\
Так как множества A, B, C представляют собой отрезки, то мы можем представить их в виде: $A[\alpha_{0},\alpha_{1}]; B[\alpha_{0}, \beta_{1}]; D[\gamma_{o},\alpha_{1}]; C[\gamma_{0},\beta_{1}].$\\
Такие значения $\alpha_{i},\beta_{i},\gamma_{i}$ были выбраны для того, чтобы выполнялось условие : $A\bigtriangleup B = C\bigtriangleup D.$\\
Тогда имеем, что $A\bigtriangleup B = C\bigtriangleup D = (\alpha_{1},\beta_{1}]$ ( условие выполняется), но $A \cap B = A = [\alpha_{0},\alpha_{1}]$, а $C = [\gamma_{0},\beta_{1}]$. По определению чисел, входящих в множество D = $[\gamma_{o},\alpha_{1}] \Rightarrow \gamma_{o} \leq \alpha_{1}$. Тогда у нас возможен случай, когда $\alpha_{0} < \gamma_{o} < \alpha_{1}$ и $A\cap B \nsubseteq C$. Приведен контрпример - значит не выполняется.

 

\begin{flushright}
\begin{large}
\textbf {Ответ: нет, не верно.}
\end{large}
\end{flushright}

{\bf 9* .} Характеристической функцией множества А называется функция:

\[ X_{A}: U \rightarrow \{0,1\} .\]\\
такая, что

\[ X_{A}(x) = \begin{cases}
1 , x\in A,\\
0 , x\notin A.\\ 
\end{cases}\]\\
Докажите, что

а)$\chi_{A\cap B}(x) = \chi_{A}(x) \cdot \chi_{B}(x);$

б)$\chi_{A\setminus B}(x) = \chi_{A}(x) - \chi_{A}(x) \cdot \chi_{B}(x);$

в)$\chi_{A\cup B}(x) = \chi_A(x) + \chi_{B}(x) - \chi_{A}(x)\cdot \chi_{B} (x);$

г)$\chi_{\overline{A}}(x) = 1 - \chi_{A}(x).$


\begin{center}
\bfseries
{\Large Решение: }
\end{center}
Основываясь на том,что функции алгебры логик аналогичны с теоретико-множественными операторами, получим:\\
a) $A \cap B = A \wedge B.$ Пользуясь условием получим, что\\
\begin{tabular}{ | c | c | c | c |}
\hline
$A$ & $B$ & $A\wedge B$ & $A \cdot B$\\ \hline
$0$ & $0$ & $0$ & $0$\\
$0$ & $1$ & $0$ & $0$\\
$1$ & $0$ & $0$ & $0$\\
$1$ & $1$ & $1$ & $1$\\
\hline 
\end{tabular}\\
Видно, что выполняется при всех (A,B)$ \Rightarrow \chi_{A\cap B}(x) = \chi_{A}(x) \cdot \chi_{B}(x)$. Доказано.\\
б) $A \setminus B = A \cap \overline{B} = A \wedge \overline{B}$\\
\begin{tabular}{ | c | c | c | c |}
\hline
$A$ & $B$ & $ A \wedge \overline{B}$ & $\chi_{A}(x) - \chi_{A}(x) \cdot \chi_{B}(x)$\\ \hline
$0$ & $0$ & $0$ & $0$\\
$0$ & $1$ & $0$ & $0$\\
$1$ & $0$ & $1$ & $1$\\
$1$ & $1$ & $0$ & $0$\\
\hline 
\end{tabular}\\
Видно, что выполняется при всех (A,B)$ \Rightarrow \chi_{A\setminus B}(x) = \chi_{A}(x) - \chi_{A}(x) \cdot \chi_{B}(x).$Доказано.\\
в) $A \cup B = A \vee B$\\
\begin{tabular}{ | c | c | c | c |}
\hline
$A$ & $B$ & $A \vee B$ & $\chi_A(x) + \chi_{B}(x) - \chi_{A}(x)\cdot \chi_{B} (x)$\\ \hline
$0$ & $0$ & $0$ & $0$\\
$0$ & $1$ & $1$ & $1$\\
$1$ & $0$ & $1$ & $1$\\
$1$ & $1$ & $1$ & $1$\\
\hline 
\end{tabular}\\
Видно, что выполняется при всех (A,B)$ \Rightarrow \chi_{A\cup B}(x) = \chi_A(x) + \chi_{B}(x) - \chi_{A}(x)\cdot \chi_{B} (x).$Доказано.\\
г) $\overline{A}$\\
\begin{tabular}{ | c | c | c |}
\hline
$A$ & $\overline{A}$ & $1 - \chi_{A}(x)$\\ \hline
$0$ & $1$ & $1$\\
$0$ & $1$ & $1$\\
$1$ & $0$ & $0$\\
$1$ & $0$ & $0$\\
\hline 
\end{tabular}\\
Видно, что выполняется при всех (A,B)$ \Rightarrow \chi_{\overline{A}}(x) = 1 - \chi_{A}(x).$Доказано.



\begin{flushright}
\begin{large}
\textbf {Ответ: доказано.}
\end{large}
\end{flushright}

{\bf 10*. }Используя формализм счетного объединения, докажите, что в любом бесконечном множестве есть счетное подмножество.

\begin{center}
\bfseries
{\Large Решение: }
\end{center}
Пусть множество {\bf B} бесконечно. Тогда оно содержит хотя бы один элемент $a_{1}$. В силу бесконечности B в нём найдется элемент $a_{2}$, отличный от $a_{1}$. Так как злементы $a_{2}$ и $a_{1}$ не исчерпывают всего множества {\bf B}, то в нём найдется элемент $a_{3}$, отличный и от $a_{2}$ и от $a_{1}$. Если уже выделено $n$ элементов$ a_{1}, a_{2},\cdots a_{n}$, то в силу бесконечности {\bf B} в нём найдётся еще один элемент, который обозначим $a_{n+1}$, отличный от всех ранее выбранных элементов. Таким образом, для каждого натурального числа $n$ можно выделить элемент $a_{n}$ из {\bf B}, причём все выделенные элементы попарно различны. Выделенные элементы образуют последовательность $ a_{1}, a_{2},\dots a_{n} \dots $. Множество её членов по определению счётно, и это множество есть часть {\bf B}.




\begin{flushright}
\begin{large}
\textbf {Ответ: доказано.}
\end{large}
\end{flushright}

\end{document}
